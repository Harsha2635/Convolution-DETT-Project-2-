\let\negmedspace\undefined
\let\negthickspace\undefined
\documentclass[journal]{IEEEtran}
\usepackage[a5paper, margin=10mm, onecolumn]{geometry}
\usepackage{tfrupee}
\usepackage{cite}
\usepackage{amsmath,amssymb,amsfonts,amsthm}
\usepackage{algorithmic}
\usepackage{graphicx}
\usepackage{textcomp}
\usepackage{xcolor}
\usepackage{txfonts}
\usepackage{listings}
\usepackage{enumitem}
\usepackage{mathtools}
\usepackage{gensymb}
\usepackage{comment}
\usepackage[breaklinks=true]{hyperref}
\usepackage{tkz-euclide} 
\usepackage[latin1]{inputenc}                                
\usepackage{color}                                            
\usepackage{array}                                            
\usepackage{longtable}                                       
\usepackage{calc}                                             
\usepackage{multirow}                                         
\usepackage{hhline}                                           
\usepackage{ifthen}                                           
\usepackage{lscape}

\begin{document}

\bibliographystyle{IEEEtran}
\vspace{3cm}

\title{Manual Convolution Analysis}
\author{EE24BTECH11004 - ANKIT JAINAR}
\maketitle

\section{Manual Convolution (with Step Signal and Rectangular Kernel)}

To start the analysis, I chose a very simple signal: the unit step function defined as:
\[
f(t) = u(t) = 
\begin{cases}
1, & t \geq 0 \\
0, & t < 0
\end{cases}
\]

The kernel we are convolving with is a rectangular pulse:
\[
h(t) =
\begin{cases}
1, & -T \leq t \leq T \\
0, & \text{otherwise}
\end{cases}
\]

The convolution formula in continuous time is:
\[
y(t) = (f * h)(t) = \int_{-\infty}^{\infty} f(\tau) h(t - \tau) d\tau
\]

Since $u(\tau) = 0$ for $\tau < 0$, we can update the limits:
\[
y(t) = \int_{0}^{\infty} h(t - \tau) d\tau
\]

Now, the function $h(t - \tau)$ is non-zero only when:
\[
-T \leq t - \tau \leq T \Rightarrow t - T \leq \tau \leq t + T
\]

Combining with the lower bound of the step function ($\tau \geq 0$), we get:
\[
\tau \in [\max(0, t - T), t + T]
\]

The integrand is 1, so evaluating the integral:
\[
y(t) = \int_{\max(0, t - T)}^{t + T} 1 \, d\tau = t + T - \max(0, t - T)
\]

\subsection{Cases}

Lets break it into different time ranges to understand the output better:

\begin{itemize}
    \item If $t < -T$: No overlap, so $y(t) = 0$
    \item If $-T \leq t < 0$: Partial overlap, area increases: $y(t) = T + t$
    \item If $0 \leq t < T$: Kernel continues to grow on step: $y(t) = T + t$
    \item If $T \leq t \leq 2T$: Full overlap: $y(t) = 2T$
    \item If $2T < t \leq 3T$: Overlap starts reducing: $y(t) = 3T - t$
    \item If $t > 3T$: No overlap again: $y(t) = 0$
\end{itemize}

The result is a trapezoidal waveform:
\[
y(t) =
\begin{cases}
0, & t < -T \\
T + t, & -T \leq t < T \\
2T, & T \leq t \leq 2T \\
3T - t, & 2T < t \leq 3T \\
0, & t > 3T
\end{cases}
\]


\end{document}

